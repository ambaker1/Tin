\documentclass{article}

% Input packages & formatting
% Packages

% Math packages
\usepackage{amsmath} % Extended math functions
\usepackage{amssymb} % Extended math symbols (loads in amsfonts)
\usepackage{bm} % Bold math symbols
\usepackage{mathtools}

% Figure packages
\usepackage{caption} % Caption formatting for university standard
\usepackage{graphicx} % includegraphics command
\usepackage{subcaption} % Subfigures
\usepackage[section]{placeins} % Place floats in section
\usepackage{wrapfig}

% Table packages
\usepackage{booktabs} % Better tables
\usepackage{bigstrut} % Merged table cells
\usepackage{longtable} % Tables which overflow into next page
\usepackage{array}
\usepackage{colortbl} % Color table cells
\usepackage{makecell}
\usepackage{multirow}

% Fonts
\usepackage{lmodern} % Use latin modern rather than computer modern. Better for font encoding.
\usepackage[T1]{fontenc} % Allow text to be searchable in output

% Other packages
\usepackage{appendix} % Appendix environment
\usepackage{nextpage} % Cleartooddpage command
%\usepackage[square,comma,sort,numbers]{natbib} % Reference formatting
\usepackage{setspace} % Line spacing
\usepackage{listings} % Display code with syntax highlighting
\usepackage{upquote} % Vertical quotes in verbatim
\usepackage{xcolor} % Colors
\usepackage{titlesec} % Header spacing
\usepackage{xparse} % for tcolorbox
\usepackage[listings]{tcolorbox} % Colored boxes for highlighting syntax
\tcbuselibrary{breakable}
\tcbuselibrary{skins}
\usepackage{enumitem} % better enumerate/itemize options
\usepackage{fancyhdr}
\usepackage{multicol}
\usepackage{ifthen}
\usepackage{xstring}

% Table of contents
\usepackage{imakeidx} % Index page
\usepackage{tocloft} % Control of table of contents
\usepackage[nottoc]{tocbibind} % Adds bibliography, table of tables, table of figures, to table of contents
\usepackage[bookmarks,linktocpage,hidelinks]{hyperref} % Hyperlinks for sections, figures, etc.

% Formatting
% Page format
\setlength{\oddsidemargin}{0.00in}  % Left side margin for odd numbered pages
\setlength{\evensidemargin}{0.00in} % Right side margin for even numbered pages
\setlength{\topmargin}{0.00in}      % Top margin
\setlength{\headheight}{0.20in}     % Header height
\setlength{\headsep}{0.20in}        % Separation between header and main text
\setlength{\topskip}{0.00in}        % Top skip
\setlength{\textwidth}{6.50in}      % Width of the text
\setlength{\textheight}{8.50in}     % Height of the text
\setlength{\footskip}{0.50in}       % Foot skip
\setlength{\parindent}{0.00in}      % First line indentation
\setlength{\parskip}{6pt}        % Space between two paragraphs

% Captions (figures, tables, etc.)
\setlength{\floatsep}{\parskip}          % Space left between floats.
\setlength{\textfloatsep}{\floatsep}   % Space between last top float
% or first bottom float and the text
\setlength{\intextsep}{\floatsep}      % Space left on top and bottom
% of an in-text float
\setlength{\abovecaptionskip}{0.1in plus 0.25in}  % Space above caption
\setlength{\belowcaptionskip}{0.1in plus 0.25in}  % Space below caption
\setlength{\captionmargin}{0.50in}     % Left/Right margin for caption
\setlength{\abovedisplayskip}{0.00in plus 0.25in} % Space before Math stuff
\setlength{\belowdisplayskip}{0.00in plus 0.25in} % Space after Math stuff
\setlength{\arraycolsep}{0.10in}       % Gap between columns of an array
\setlength{\jot}{0.10in}                % Gap between multiline equations
\setlength{\itemsep}{0.10in}           % Space between successive items

% Counters (no section numbering)
\setcounter{tocdepth}{3}
\setcounter{secnumdepth}{0}

% Spacing
\setstretch{1.5}

\titlespacing*{\section}{0cm}{6pt}{6pt}[0cm]
\titlespacing*{\subsection}{0cm}{6pt}{6pt}[0cm]
\titlespacing*{\subsubsection}{0cm}{6pt}{6pt}[0cm]

\titleformat{\section}
{\sffamily\huge}{}{0pt}{\titlerule\vspace{-0.2cm}}
\titleformat{\subsection}
{\sffamily\itshape\Large}{}{0pt}{}

% Macro for syntax
\newtcolorbox{syntax}{
    size=small,
    sharp corners,
    colframe=black,
    colback=yellow,
    fontupper=\bfseries\ttfamily
}

% Macro for argument table
\newenvironment{args}{
    \begin{tabular}{>{\bfseries\ttfamily}p{0.25\linewidth} p{0.69\linewidth}}
    }{
    \end{tabular}\par
    \vspace{0.5\baselineskip}
}

% Note: Requires packages "listing", "xcolor", and "textcomp"
\lstdefinelanguage{verbatim}{
    basicstyle=\ttfamily\small,
    xleftmargin=9pt,
    xrightmargin=9pt,
    columns=fullflexible,
    keepspaces=true,
    breaklines=true
}

% Example code
\AtBeginDocument{
\newtcolorbox[blend into=listings]{example}[2][]{
    colback=blue!3!white,
    colframe=black,
    colbacktitle=blue!15!white,
    coltitle=black,
    sharp corners,
    enhanced,
    breakable,
    size=small,
    before upper={
        \setstretch{1.0}\lstset{language=verbatim}\vspace{3pt}\textsf{\textit{Code:}}
    },
    subtitle style={
        colback=blue!20!white,
        fonttitle=\sffamily
    },
    before lower={
        \setstretch{1.0}\lstset{language=verbatim}\vspace{3pt}\textsf{\textit{Output:}}
    },
    fonttitle=\sffamily,
    title={#2},
    #1
}
}

% Links to sub and subsub commands - optional boolean argument, default true. if false, only displays subcmd.

% Commands (and command ensembles)
\newcommand{\command}[1]{\protect\hypertarget{#1}{#1}\index{#1}}
\newcommand{\subcommand}[2]{\protect\hypertarget{#1 #2}{#1 #2}\index{#1!#2}}
\newcommand{\cmdlink}[1]{\protect\hyperlink{#1}{\textit{#1}}}
\newcommand{\subcmdlink}[3][1]{\protect\hyperlink{#2 #3}{\ifnum#1=1\relax\textit{#2 #3}\else\textit{#3}\fi}}

% Methods (first arg is class)
\newcommand{\method}[2]{\protect\hypertarget{$#1Obj #2}{\$#1Obj #2}\index{#1 methods!#2}}
\newcommand{\methodlink}[3][1]{\protect\hyperlink{$#2Obj #3}{\ifnum#1=1\relax\textit{\$#2Obj #3}\else\textit{#3}\fi}}

% Macros for figure/table names
\newcommand{\fig}{\figurename\ }
\newcommand{\figs}{\figurename s }
\newcommand{\tbl}{\tablename\ }
\newcommand{\tbls}{\tablename s }
\newcommand{\eq}{Eq. }
\newcommand{\eqs}{Eqs. }
\renewcommand{\lstlistingname}{Example}% Listing -> Example
\renewcommand{\lstlistlistingname}{List of \lstlistingname s}% List of Listings -> List of Examples
\newcommand{\ex}{Example }
\newcommand{\exs}{Examples }
\newcommand{\var}[1]{\texttt{\textbf{\$#1}}}

% Header/footer
\renewcommand{\headrulewidth}{0pt}

% Changes to hyperlinks (URLs)
\renewcommand\UrlFont{\color{blue}\rmfamily}

% New column type 
% https://tex.stackexchange.com/questions/75717/how-can-i-mix-itemize-and-tabular-environments
\newcolumntype{L}{>{\labelitemi~~}l<{}}
\renewcommand{\cleartooddpage}[1][]{\ignorespaces} % single side
\newcommand{\caret}{$^\wedge$}

% Other macros
\renewcommand{\^}[1]{\textsuperscript{#1}}
\renewcommand{\_}[1]{\textsubscript{#1}}

\title{\Huge Tin: A Tcl Package Manager\\\small Version 0.3}
\author{Alex Baker\\\small\hyperlink{https://github.com/ambaker1/tin}{https://github.com/ambaker1/tin}}
\date{\small\today}
\begin{document}
\maketitle
\clearpage
\section{Installing Packages}
The command \cmdlink{tin install} installs packages directly from GitHub.
\begin{syntax}
\command{tin install} \$name <{}<-exact> \$version> <\$reqs ...>
\end{syntax}
\begin{args}
\$name & Package name. \\
-exact & Option to install exact version. \\
\$version & Package version. \\
\$reqs ... & Package version requirements, mutually exclusive with -exact option.
\end{args}

\begin{example}{Upgrading Tin}
\begin{lstlisting}
package require tin
tin install tin
\end{lstlisting}
\end{example}

The available packages in the Tin database can be queried with the command \cmdlink{tin packages}, and the versions for each package in ascending order can be queried with the command \cmdlink{tin versions}. If version requirements are specified in \cmdlink{tin versions}, it will filter using the rules of \textit{package vsatisfies}.

\begin{syntax}
\command{tin packages} <\$pattern>
\end{syntax}
\begin{args}
\$pattern & Optional ``glob'' pattern, default ``\texttt{*}'', or all packages.
\end{args}

\begin{syntax}
\command{tin versions} \$name <{}<-exact> \$version> <\$reqs ...>
\end{syntax}
\begin{args}
\$name & Package name. \\
-exact & Option to install exact version. \\
\$version & Package version. \\
\$reqs ... & Package version requirements, mutually exclusive with -exact option.
\end{args}

\clearpage
\section{Modifying the Tin Package Database}
The ``tinlist.tcl'' file in the Tin installation initializes the Tin database, which can be modified within your local instance of Tcl with the commands \cmdlink{tin add} and \cmdlink{tin remove}.

\begin{syntax}
\command{tin add} \$name \$version \$repo \$tag \$installer
\end{syntax}
\begin{args}
\$name & Package name. \\
\$version & Package version. \\
\$repo & Github repository URL. \\
\$tag & Github release tag for version. \\
\$installer & Installer file path in repo.
\end{args}

\begin{syntax}
\command{tin remove} \$name \$version ...
\end{syntax}
\begin{args}
\$name & Package name. \\
\$version ... & Package version(s) to remove.
\end{args}

\begin{example}{Adding a package to the Tin database}
\begin{lstlisting}
package require tin
tin add foo 1.0 https://github.com/username/foo v1.0 install_foo.tcl
\end{lstlisting}
\end{example}

\clearpage
\section{Loading and Importing Packages}
Tin also provides advanced tools for loading and importing packages.
The command \cmdlink{tin require} is similar to the Tcl command \textit{package require}, but with the added feature that if the package is missing, it will try to install it with \cmdlink{tin install}.
The command \cmdlink{tin import} additionally handles most use-cases of \textit{namespace import}. 
Both \cmdlink{tin require} and \cmdlink{tin import} return the version number of the package imported.
\begin{syntax}
\command{tin require} \$name <{}<-exact> \$version> <\$reqs ...>
\end{syntax}
\begin{args}
\$name & Package name. \\
-exact & Option to install exact version. \\
\$version & Package version. \\
\$reqs ... & Package version requirements, mutually exclusive with -exact option.
\end{args}
\begin{syntax}
\command{tin import} <-force> <\$patterns from> \$name <{}<-exact> \$version> <\$reqs ...> <as \$ns>
\end{syntax}
\begin{args}
-force & Option to overwrite existing commands. \\
\$patterns & Commands to import, or ``glob'' patterns, default ``\texttt{*}'', or all commands. \\
\$name & Package name. \\
-exact & Option to install exact version. \\
\$version & Package version. \\
\$reqs ... & Package version requirements, mutually exclusive with -exact option. \\
\$ns & Namespace to import into. Default global namespace, or ``\texttt{::}''.
\end{args}

\begin{example}{Importing all commands package ``foo''}
\begin{lstlisting}
package require tin
tin import foo 1.0
\end{lstlisting}
\end{example}

\clearpage
\section{Writing Installation Files}
Tin also provides utilities to simplify writing installation files.
The command \cmdlink{tin mkdir} creates a library directory to install a package in, relative to the base library directory (\cmdlink{tin library}).
The command \cmdlink{tin depend} checks if a package is installed without loading the package. 
If a package is not installed, it will try to install it with \cmdlink{tin install}.
\begin{syntax}
\command{tin mkdir} <-force> \$path
\end{syntax}
\begin{args}
-force & Option to create fresh library directory (deletes existing folder). \\
\$path & Relative path to \cmdlink{tin library}.
\end{args}
\begin{syntax}
\command{tin library} <\$path>
\end{syntax}
\begin{args}
\$path & Base library directory. Default returns the current base library directory. 
\end{args}
\begin{syntax}
\command{tin depend} \$name <{}<-exact> \$version> <\$reqs ...>
\end{syntax}
\begin{args}
\$name & Package name. \\
-exact & Option to install exact version. \\
\$version & Package version. \\
\$reqs ... & Package version requirements, mutually exclusive with -exact option. 
\end{args}

See the example installation file for a package ``foo'' that requires the package ``bar 1.2'':
\begin{example}{Example installation file ``install\textunderscore{}foo.tcl''}
\begin{lstlisting}
package require tin
tin depend bar 1.2
set dir [tin mkdir -force foo]
file copy README.md $dir
file copy LICENSE $dir
file copy lib/bar.tcl $dir
file copy lib/pkgIndex.pdf $dir
\end{lstlisting}
\end{example}
\end{document}

