\documentclass{article}

% Input packages & formatting
% Packages

% Math packages
\usepackage{amsmath} % Extended math functions
\usepackage{amssymb} % Extended math symbols (loads in amsfonts)
\usepackage{bm} % Bold math symbols
\usepackage{mathtools}

% Figure packages
\usepackage{caption} % Caption formatting for university standard
\usepackage{graphicx} % includegraphics command
\usepackage{subcaption} % Subfigures
\usepackage[section]{placeins} % Place floats in section
\usepackage{wrapfig}

% Table packages
\usepackage{booktabs} % Better tables
\usepackage{bigstrut} % Merged table cells
\usepackage{longtable} % Tables which overflow into next page
\usepackage{array}
\usepackage{colortbl} % Color table cells
\usepackage{makecell}
\usepackage{multirow}

% Fonts
\usepackage{lmodern} % Use latin modern rather than computer modern. Better for font encoding.
\usepackage[T1]{fontenc} % Allow text to be searchable in output

% Other packages
\usepackage{appendix} % Appendix environment
\usepackage{nextpage} % Cleartooddpage command
%\usepackage[square,comma,sort,numbers]{natbib} % Reference formatting
\usepackage{setspace} % Line spacing
\usepackage{listings} % Display code with syntax highlighting
\usepackage{upquote} % Vertical quotes in verbatim
\usepackage{xcolor} % Colors
\usepackage{titlesec} % Header spacing
\usepackage{xparse} % for tcolorbox
\usepackage[listings]{tcolorbox} % Colored boxes for highlighting syntax
\tcbuselibrary{breakable}
\tcbuselibrary{skins}
\usepackage{enumitem} % better enumerate/itemize options
\usepackage{fancyhdr}
\usepackage{multicol}
\usepackage{ifthen}
\usepackage{xstring}

% Table of contents
\usepackage{imakeidx} % Index page
\usepackage{tocloft} % Control of table of contents
\usepackage[nottoc]{tocbibind} % Adds bibliography, table of tables, table of figures, to table of contents
\usepackage[bookmarks,linktocpage,hidelinks]{hyperref} % Hyperlinks for sections, figures, etc.

% Formatting
% Page format
\setlength{\oddsidemargin}{0.00in}  % Left side margin for odd numbered pages
\setlength{\evensidemargin}{0.00in} % Right side margin for even numbered pages
\setlength{\topmargin}{0.00in}      % Top margin
\setlength{\headheight}{0.20in}     % Header height
\setlength{\headsep}{0.20in}        % Separation between header and main text
\setlength{\topskip}{0.00in}        % Top skip
\setlength{\textwidth}{6.50in}      % Width of the text
\setlength{\textheight}{8.50in}     % Height of the text
\setlength{\footskip}{0.50in}       % Foot skip
\setlength{\parindent}{0.00in}      % First line indentation
\setlength{\parskip}{6pt}        % Space between two paragraphs

% Captions (figures, tables, etc.)
\setlength{\floatsep}{\parskip}          % Space left between floats.
\setlength{\textfloatsep}{\floatsep}   % Space between last top float
% or first bottom float and the text
\setlength{\intextsep}{\floatsep}      % Space left on top and bottom
% of an in-text float
\setlength{\abovecaptionskip}{0.1in plus 0.25in}  % Space above caption
\setlength{\belowcaptionskip}{0.1in plus 0.25in}  % Space below caption
\setlength{\captionmargin}{0.50in}     % Left/Right margin for caption
\setlength{\abovedisplayskip}{0.00in plus 0.25in} % Space before Math stuff
\setlength{\belowdisplayskip}{0.00in plus 0.25in} % Space after Math stuff
\setlength{\arraycolsep}{0.10in}       % Gap between columns of an array
\setlength{\jot}{0.10in}                % Gap between multiline equations
\setlength{\itemsep}{0.10in}           % Space between successive items

% Counters (no section numbering)
\setcounter{tocdepth}{3}
\setcounter{secnumdepth}{0}

% Spacing
\setstretch{1.5}

\titlespacing*{\section}{0cm}{6pt}{6pt}[0cm]
\titlespacing*{\subsection}{0cm}{6pt}{6pt}[0cm]
\titlespacing*{\subsubsection}{0cm}{6pt}{6pt}[0cm]

\titleformat{\section}
{\sffamily\huge}{}{0pt}{\titlerule\vspace{-0.2cm}}
\titleformat{\subsection}
{\sffamily\itshape\Large}{}{0pt}{}

% Macro for syntax
\newtcolorbox{syntax}{
    size=small,
    sharp corners,
    colframe=black,
    colback=yellow,
    fontupper=\bfseries\ttfamily
}

% Macro for argument table
\newenvironment{args}{
    \begin{tabular}{>{\bfseries\ttfamily}p{0.25\linewidth} p{0.69\linewidth}}
    }{
    \end{tabular}\par
    \vspace{0.5\baselineskip}
}

% Note: Requires packages "listing", "xcolor", and "textcomp"
\lstdefinelanguage{verbatim}{
    basicstyle=\ttfamily\small,
    xleftmargin=9pt,
    xrightmargin=9pt,
    columns=fullflexible,
    keepspaces=true,
    breaklines=true
}

% Example code
\AtBeginDocument{
\newtcolorbox[blend into=listings]{example}[2][]{
    colback=blue!3!white,
    colframe=black,
    colbacktitle=blue!15!white,
    coltitle=black,
    sharp corners,
    enhanced,
    breakable,
    size=small,
    before upper={
        \setstretch{1.0}\lstset{language=verbatim}\vspace{3pt}\textsf{\textit{Code:}}
    },
    subtitle style={
        colback=blue!20!white,
        fonttitle=\sffamily
    },
    before lower={
        \setstretch{1.0}\lstset{language=verbatim}\vspace{3pt}\textsf{\textit{Output:}}
    },
    fonttitle=\sffamily,
    title={#2},
    #1
}
}

% Links to sub and subsub commands - optional boolean argument, default true. if false, only displays subcmd.

% Commands (and command ensembles)
\newcommand{\command}[1]{\protect\hypertarget{#1}{#1}\index{#1}}
\newcommand{\subcommand}[2]{\protect\hypertarget{#1 #2}{#1 #2}\index{#1!#2}}
\newcommand{\cmdlink}[1]{\protect\hyperlink{#1}{\textit{#1}}}
\newcommand{\subcmdlink}[3][1]{\protect\hyperlink{#2 #3}{\ifnum#1=1\relax\textit{#2 #3}\else\textit{#3}\fi}}

% Methods (first arg is class)
\newcommand{\method}[2]{\protect\hypertarget{$#1Obj #2}{\$#1Obj #2}\index{#1 methods!#2}}
\newcommand{\methodlink}[3][1]{\protect\hyperlink{$#2Obj #3}{\ifnum#1=1\relax\textit{\$#2Obj #3}\else\textit{#3}\fi}}

% Macros for figure/table names
\newcommand{\fig}{\figurename\ }
\newcommand{\figs}{\figurename s }
\newcommand{\tbl}{\tablename\ }
\newcommand{\tbls}{\tablename s }
\newcommand{\eq}{Eq. }
\newcommand{\eqs}{Eqs. }
\renewcommand{\lstlistingname}{Example}% Listing -> Example
\renewcommand{\lstlistlistingname}{List of \lstlistingname s}% List of Listings -> List of Examples
\newcommand{\ex}{Example }
\newcommand{\exs}{Examples }
\newcommand{\var}[1]{\texttt{\textbf{\$#1}}}

% Header/footer
\renewcommand{\headrulewidth}{0pt}

% Changes to hyperlinks (URLs)
\renewcommand\UrlFont{\color{blue}\rmfamily}

% New column type 
% https://tex.stackexchange.com/questions/75717/how-can-i-mix-itemize-and-tabular-environments
\newcolumntype{L}{>{\labelitemi~~}l<{}}
\newcommand{\version}{0.4.4}

\renewcommand{\cleartooddpage}[1][]{\ignorespaces} % single side
\newcommand{\caret}{$^\wedge$}

\title{\Huge Tin: A Tcl Package Manager\\\small Version \version}
\author{Alex Baker\\\small\url{https://github.com/ambaker1/Tin}}
\date{\small\today}
\begin{document}
\maketitle
\begin{abstract}
\begin{center}
Tin is a Tcl package that installs Tcl packages and their dependencies directly from GitHub. 
Additionally, Tin provides various utilities for Tcl package ecosystem development.
\end{center}
\end{abstract}
\clearpage
\section{The Official Tin List}
The Official Tin List is a database of Tcl packages that can be installed directly from GitHub. \\
As of Tin version \version, these are the packages in the Official Tin List:
\subsubsection{Auto-Tin Packages}
\begin{tabular}{llll}
Package & Repo & File & Version Requirements \\
\midrule
flytrap & \url{https://github.com/ambaker1/flytrap} & install.tcl & 0- \\
tda & \url{https://github.com/ambaker1/Tda} & install.tcl & 0.1.1- \\
tin & \url{https://github.com/ambaker1/Tin} & install.tcl & 0 \\
tintest & \url{https://github.com/ambaker1/Tin-Test} & install.tcl & 0- \\
vutil & \url{https://github.com/ambaker1/vutil} & install.tcl & 0.1.1- \\
wob & \url{https://github.com/ambaker1/wob} & install.tcl & 0.1.3- \\
\bottomrule
\end{tabular}

\clearpage
\section{Adding to the Tin List}
Tin installs packages based on entries in the ``Tin List'', which can be added to in the current session with the commands \cmdlink{tin add}.
\begin{syntax}
\command{tin add} <-tin> \$name \$version \$repo \$tag \$file
\end{syntax}
\begin{args}
-tin & Option to add an entry to the Tin List. Default. \\
\$name & Package name. \\
\$version & Package version. \\
\$repo & Github repository URL. \\
\$tag & Github release tag for version.  \\
\$file & Installer file path in repo. 
\end{args}
\begin{example}{Adding entries to the Tin List}
\begin{lstlisting}
tin add foo 1.0 https://github.com/username/foo v1.0 install_foo.tcl
tin add foo 1.1 https://github.com/username/foo v1.1 install_foo.tcl
tin add foo 1.2 https://github.com/username/foo v1.2 install_foo.tcl
tin add foo 1.2.1 https://github.com/username/foo v1.2.1 install_foo.tcl
tin add foo 1.2.2 https://github.com/username/foo v1.2.2 install_foo.tcl
tin add foo 2a0 https://github.com/username/foo v2a0 install_foo.tcl
tin add foo 2.0 https://github.com/username/foo v2.0 install_foo.tcl
tin add foo 2.0.1 https://github.com/username/foo v2.0.1 install_foo.tcl
\end{lstlisting}
\end{example}
Clearly, this is a bit verbose... and repetitive...
\clearpage
\section{Auto-Tin Configuration}
To streamline the process of populating the Tin List, simply use the command \cmdlink{tin add -auto} to specify an Auto-Tin configuration.
An Auto-Tin configuration specifies a template for Tin entries, and this template is used in conjunction with GitHub release tags to automatically populate the Tin List.
\begin{syntax}
\command{tin add -auto} \$name \$repo \$file <{}<-exact> \$version> <\$reqs ...>
\end{syntax}
\begin{args}
-auto & Option to add an Auto-Tin configuration to the Tin List. \\
\$name & Package name. \\
\$repo & Github repository URL. \\
\$file & Installer file path in repo. \\
\$version & Package version (-exact specifies exact version). \\
\$reqs ... & Package version requirements, mutually exclusive with -exact option. 
\end{args}
\subsection{Auto-Tin Tag Pattern}
In order for an Auto-Tin configuration to work, the GitHub repository must have release tags corresponding directly with the package versions, such as ``v1.2.3''.

To be specific, version release tags must match the Auto-Tin tag pattern:
\begin{lstlisting}[language=verbatim]
^v(0|[1-9]\d*)(\.(0|[1-9]\d*))*([ab](0|[1-9]\d*)(\.(0|[1-9]\d*))*)?$
\end{lstlisting}
Note that this is not the same as ``SemVer'' (\url{https://semver.org/}), which is the standard version number format. 
This is because the Auto-Tin tag pattern specifically matches all version numbers compatible with Tcl, with a prefix of ``v''. 
Most notably, the format for alpha and beta versions is different in Tcl, where ``a'' and ``b'' replace one of the periods (e.g. ``v2a0''), and effectively represent ``.-2.'' and ``.-1.'', respectively.
\clearpage
\subsection{Fetching and Auto-Fetching}
If a package is configured as an Auto-Tin package, the Tin List can be automatically populated with versions available for installation with the command \cmdlink{tin fetch}. 
If the -all option is specified, it will return a dictionary of package names and versions added.
Otherwise, it will simply return a list of the versions added for the package. 
\begin{syntax}
\command{tin fetch} \$name <\$pattern> \\
tin fetch -all <\$names>
\end{syntax}
\begin{args}
\$name & Package name. \\
\$pattern & Version number glob pattern for \textit{git ls-remote}. Default ``\texttt{*}'', or all versions. \\
-all & Option to fetch all available versions. \\
\$names & List of package names. Default all Auto-Tin packages.
\end{args}
Additionally, by default, Tin auto-fetches on the fly when it cannot find Tin entries satisfying version requirements. 
This can be toggled on/off with \cmdlink{tin auto}, which also returns the current auto-fetch status.
\begin{syntax}
\command{tin auto} <\$toggle>
\end{syntax}
\begin{args}
\$toggle & Boolean, whether to auto-fetch. Default true.
\end{args}
\begin{example}{Adding an Auto-Tin configuration and fetching Tin entries}
\begin{lstlisting}
tin add -auto foo https://github.com/username/foo install_foo.tcl
tin fetch foo; # fetches all entries from GitHub
\end{lstlisting}
\end{example}

\clearpage
\section{Removing from the Tin List}
Both Tin entries and Auto-Tin configurations can be removed from the Tin List in a session with the command \cmdlink{tin remove}. 
\begin{syntax}
\command{tin remove} \$name ... \\
tin remove -tin \$name <\$version> <\$repo> \\
tin remove -auto \$name <\$repo> <\$file> 
\end{syntax}
\begin{args}
\$name & Package name. \\
-tin & Option to only remove from Tin. \\
-auto & Option to only remove from Auto-Tin. \\
\$version & Package version to remove (optional, default all versions). \\
\$repo & Repository to remove (optional, default all repositories). \\
\$file & Installer file path to remove (optional, default all installer files).
\end{args}
\subsection{Clearing the Tin List}
The command \cmdlink{tin clear} removes all entries and Auto-Tin configurations from the Tin List.
\begin{syntax}
\command{tin clear}
\end{syntax}


\clearpage
\section{Saving and Resetting the Tin List}
The state of the Tin List can be saved for future sessions with \cmdlink{tin save}, and reset to default or factory settings with \cmdlink{tin reset}. 
Note that \cmdlink{tin save} saves to a hidden user-config file located in the user's home directory.
\begin{syntax}
\command{tin save}
\end{syntax}
\begin{syntax}
\command{tin reset} <-soft> \\
tin reset -hard
\end{syntax}
\begin{args}
-soft & Option to reset to user settings (default). \\
-hard & Option to reset to factory settings.
\end{args}

\begin{example}{Saving changes to the Tin List}
\begin{lstlisting}
tin reset -hard
tin add foo 1.0 https://github.com/username/foo v1.0 install_foo.tcl
tin save
\end{lstlisting}
\tcblower

\textit{"$\sim$/.tinlist.tcl" :}
\begin{lstlisting}
tin add foo 1.0 https://github.com/username/foo v1.0 install_foo.tcl
\end{lstlisting}
\end{example}


\clearpage
\section{Available Packages}
The available packages in the Tin List can be queried with the command \cmdlink{tin packages}.
\begin{syntax}
\command{tin packages} <\$pattern> \\
tin packages -tin <\$pattern> \\
tin packages -auto <\$pattern>
\end{syntax}
\begin{args}
\$pattern & Optional ``glob'' pattern, default ``\texttt{*}'', or all packages. \\
-tin & Option to search only for Tin packages. \\
-auto & Option to search only for Auto-Tin packages. 
\end{args}
\begin{example}{Getting available packages}
\begin{lstlisting}
set allPackages [tin packages]
set loadedPackages [tin packages -tin]
set autoPackages [tin packages -auto]
\end{lstlisting}
\end{example}
\subsection{Available Package Versions}
A list of available versions for a Tin package that satisfy version requirements can be queried with the command \cmdlink{tin versions}.
Similarly, the command \cmdlink{tin available} returns the version that would be installed with \cmdlink{tin install}, or blank if none can be found.
\begin{syntax}
\command{tin versions} \$name <{}<-exact> \$version> <\$reqs ...>
\end{syntax}
\begin{syntax}
\command{tin available} \$name <{}<-exact> \$version> <\$reqs ...>
\end{syntax}
\begin{args}
\$name & Package name. \\
\$version & Package version (-exact specifies exact version). \\
\$reqs ... & Package version requirements, mutually exclusive with -exact option.
\end{args}
\clearpage
\section{Accessing the Tin List}
The command \cmdlink{tin get} queries basic information about Tin, and returns blank if the requested entry does exist. 
Similar to \cmdlink{tin add} and \cmdlink{tin remove}, it has two forms, one for querying Tin packages and one for querying Auto-Tin packages. 
Returns a dictionary associated with the supplied arguments.
\begin{syntax}
\command{tin get} <-tin> \$name <\$version> <\$repo> \\
tin get -auto \$name <\$repo> <\$file>
\end{syntax}
\begin{args}
\$name & Package name. \\
-tin & Option to query Tin packages (default). \\
-auto & Option to query Auto-Tin packages. \\
\$version & Package version.  \\
\$repo & Github repository URL. \\
\$file & Installer file path in repo.
\end{args}
\begin{example}{Getting info from the Tin List}
\begin{lstlisting}
package require tin
tin add foo 1.0 https://github.com/username/foo v1.0 install_foo.tcl
puts [tin get foo]
\end{lstlisting}
\tcblower
\begin{lstlisting}
1.0 {https://github.com/username/foo {v1.0 install_foo.tcl}}
\end{lstlisting}
\end{example}

\clearpage
\section{Installing and Uninstalling Packages}
The command \cmdlink{tin install} installs packages directly from GitHub, and returns the version installed.

The command \cmdlink{tin depend} installs packages only if they are not installed, and returns the version number installed (useful for dependencies in installation scripts).

The command \cmdlink{tin uninstall} uninstalls packages (as long as they are in the Tin List), and returns blank if successful, or error if it was unsuccessful in uninstalling the package. 

\begin{syntax}
\command{tin install} \$name <{}<-exact> \$version> <\$reqs ...>
\end{syntax}
\begin{syntax}
\command{tin depend} \$name <{}<-exact> \$version> <\$reqs ...>
\end{syntax}
\begin{syntax}
\command{tin uninstall} \$name <{}<-exact> \$version> <\$reqs ...>
\end{syntax}
\begin{args}
\$name & Package name. \\
\$version & Package version (-exact specifies exact version). \\
\$reqs ... & Package version requirements, mutually exclusive with -exact option.
\end{args}

By default, uninstalling a package simply deletes the library folder associated with the package.
However, if a ``pkgUninstall.tcl'' file is located within the package folder, it will run that file instead, with the variable \texttt{\$dir} set to the package library folder path, similar to how ``pkgIndex.tcl'' files work.

\begin{example}{Complex uninstall file ``pkgUninstall.tcl''}
\begin{lstlisting}
set bindir [file dirname [info nameofexecutable]]
file delete [file join $bindir foo.bat]; # delete file in the bin directory
file delete -force $dir; # Clean up package
\end{lstlisting}
\end{example}
\clearpage
\section{Upgrading Packages}
Upgrading packages involves first installing the new version with \cmdlink{tin install}, then uninstalling the old version with \cmdlink{tin uninstall}.
The command \cmdlink{tin check} returns an upgrade list of available minor and patch upgrades for packages.
The command \cmdlink{tin upgrade} checks for available upgrades with \cmdlink{tin check}, and then upgrades the packages, returning the upgrade list. 
If there is no upgrade available, the upgrade list will be empty. 
If the -all option is specified, the format of the upgrade list is ``\texttt{name \{old new ...\} ...}''. 
Otherwise, the format of the upgrade list is ``\texttt{old new}''.
\begin{syntax}
\command{tin check} \$name <{}<-exact> \$version> <\$reqs ...> \\
tin check -all <\$names> 
\end{syntax}
\begin{syntax}
\command{tin upgrade} \$name <{}<-exact> \$version> <\$reqs ...> \\
tin upgrade -all <\$names> 
\end{syntax}
\begin{args}
-all & Option to look for upgrades in all installed major versions of the packages. \\
\$names & Package names. Default searches all packages in the Tin List.\\
\$name & Package name. \\
\$version & Package version (-exact specifies exact version). \\
\$reqs ... & Package version requirements, mutually exclusive with -exact option.
\end{args}
\begin{example}{Upgrading Tin}
\begin{lstlisting}
# Upgrade Tin
package require tin
tin upgrade tin
\end{lstlisting}
\end{example}

\clearpage
\section{Loading and Importing Packages}
The command \cmdlink{tin require} is similar to the Tcl command \textit{package require}, but with the added feature that if the package is missing, it will try to install it with \cmdlink{tin install}.

The command \cmdlink{tin import} additionally handles most use-cases of \textit{namespace import}. 
Both \cmdlink{tin require} and \cmdlink{tin import} return the version number of the package imported.
\begin{syntax}
\command{tin require} \$name <{}<-exact> \$version> <\$reqs ...>
\end{syntax}
\begin{args}
\$name & Package name. \\
\$version & Package version (-exact specifies exact version). \\
\$reqs ... & Package version requirements, mutually exclusive with -exact option.
\end{args}
\begin{syntax}
\command{tin import} <-force> <\$patterns from> \$name <{}<-exact> \$version> <\$reqs ...> <as \$ns>
\end{syntax}
\begin{args}
-force & Option to overwrite existing commands. \\
\$patterns & Commands to import, or ``glob'' patterns, default ``\texttt{*}'', or all commands. \\
\$name & Package name. \\
\$version & Package version (-exact specifies exact version). \\
\$reqs ... & Package version requirements, mutually exclusive with -exact option. \\
\$ns & Namespace to import into. Default global namespace, or ``\texttt{::}''.
\end{args}

\begin{example}{Importing all commands package ``foo''}
\begin{lstlisting}
package require tin
tin import foo 1.0
\end{lstlisting}
\end{example}
\clearpage
\section{Generic Package Utilities}
Tin works alongside the Tcl \textit{package} commands, and provides a few package utility commands that do not interface with the Tin List.
\subsection{Check if a Package is Installed}
The command \cmdlink{tin installed} returns the package version number that would be loaded with \textit{package require}, or blank if the package is not installed. 
If there is no package version in the Tcl package database satisfying the requirements, it will call the \cmdlink{package unknown} script to load \textit{package ifneeded} statements from ``pkgIndex.tcl'' files, just like what \textit{package require} does, but without loading the package.
\begin{syntax}
\command{tin installed} \$name <{}<-exact> \$version> <\$reqs ...>
\end{syntax}
\begin{args}
\$name & Package name. \\
\$version & Package version (-exact specifies exact version). \\
\$reqs ... & Package version requirements, mutually exclusive with -exact option.
\end{args}
\subsection{Unload a Package/Namespace}
The command \cmdlink{tin forget} is short-hand for both \textit{package forget} and \textit{namespace delete}, but it will not throw an error if there is no namespace corresponding with the package name.
It is especially useful for reloading packages within an instance of Tcl.
\begin{syntax}
\command{tin forget} \$name ...
\end{syntax}
\begin{args}
\$name & Package name. 
\end{args}
\begin{example}{Loading and reloading a package}
\begin{lstlisting}
package require foo
tin forget foo
package require foo
\end{lstlisting}
\end{example}
\clearpage
\section{Utilities for Package Development}
In addition to commands for installing and loading packages, Tin provides a few commands intended to help in writing installation and build files for your packages.
\subsection{Creating Package Directories}
The command \cmdlink{tin mkdir} creates a library directory to install a package in, with a normalized naming convention that allows it to be uninstalled easily with \cmdlink{tin uninstall}. 
\begin{syntax}
\command{tin mkdir} <-force> <\$basedir> \$name \$version
\end{syntax}
\begin{args}
-force & Option to create fresh library directory (deletes existing folder). \\
\$basedir & Base directory, default one folder up from the Tcl library folder. \\
\$name & Package name. \\
\$version & Package version.
\end{args}

See the example installation file for a package ``foo'' that requires the package ``bar 1.2'', and installs in library folder ``foo-1.0''.
\begin{example}{Example file ``install\textunderscore{}foo.tcl''}
\begin{lstlisting}
package require tin
tin depend bar 1.2
set dir [tin mkdir -force foo 1.0]
file copy README.md $dir
file copy LICENSE $dir
file copy lib/bar.tcl $dir
file copy lib/pkgIndex.pdf $dir
\end{lstlisting}
\end{example}

\clearpage
\subsection{Building Library Files from Source with Configuration Variable Substitution}
The command \cmdlink{tin bake} takes an input text file, and writes an output text file after substitution of configuration variables such as \texttt{@VERSION@}.
This is especially helpful for ensuring that the package version is consistent across the entire project.
If a source directory is used as input, it will batch bake all ``.tin'' files. 

\begin{syntax}
\command{tin bake} \$src \$target \$config \\
tin bake \$src \$target \$varName \$value ...
\end{syntax}
\begin{args}
\$src & Source file, or directory with ``.tin'' files. \\
\$target & Target file, or directory to write ``.tcl'' files to. \\
\$config & Dictionary of config variable names and values. Config variables must be uppercase alphanumeric. \\
\$varName \$value ... & Config variable names and values. Mutually exclusive with \texttt{\$config}.
\end{args}

See below for an example of how \cmdlink{tin bake} can be used to automatically update a ``pkgIndex.tcl'' file:

\begin{example}{Building a ``pkgIndex.tcl'' file}
\begin{lstlisting}
package require tin
tin bake pkgIndex.tin pkgIndex.tcl {VERSION 1.0}
\end{lstlisting}
\tcblower

\textit{"pkgIndex.tin" :}
\begin{lstlisting}
package ifneeded foo @VERSION@ [list source [file join $dir foo.tcl]]
\end{lstlisting}
\textit{"pkgIndex.tcl" :}
\begin{lstlisting}
package ifneeded foo 1.0 [list source [file join $dir foo.tcl]]
\end{lstlisting}
\end{example}
\clearpage
\subsection{Basic Unit Testing}
The command \cmdlink{tin assert} is intended for basic unit testing of Tcl scripts (or sanitizing user input).
It throws an error if the statement is false, otherwise it simply returns nothing and the script continues.
For more advanced unit testing, the built-in \textit{tcltest} package is recommended.
\begin{syntax}
\command{tin assert} \$expr <\$message>
\end{syntax}
\begin{args}
\$expr & Boolean expression to evaluate. \\
\$message & Optional error message to add context to assertion error. Default blank.
\end{args}
\begin{example}{Asserting values and types}
\begin{lstlisting}
tin assert {2 + 2 == 4}; # Asserts that math works
tin assert {[string is double -strict 5.0]}; # Asserts that 5.0 is indeed a number
tin assert {[string is integer "hello world"]}; # This is false
\end{lstlisting} 
\tcblower
\begin{lstlisting}
assertion error
    while executing
"tin assert {[string is integer "hello world"]}"
\end{lstlisting}
\end{example}

\begin{example}{Provide context to invalid procedure input}
\begin{lstlisting}
proc subtract {x y} {
    tin assert {$x > $y} "x must be greater than y"
    expr {$x - $y}
}
subtract 2.0 3.0
\end{lstlisting}
\tcblower
\begin{lstlisting}
assertion error: x must be greater than y
    while executing
"tin assert {$x > $y} "x must be greater than y""
    (procedure "subtract" line 2)
    invoked from within
"subtract 2.0 3.0"
\end{lstlisting}
\end{example}
\end{document}



