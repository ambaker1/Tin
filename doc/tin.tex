\documentclass{article}

% Input packages & formatting
% Packages

% Math packages
\usepackage{amsmath} % Extended math functions
\usepackage{amssymb} % Extended math symbols (loads in amsfonts)
\usepackage{bm} % Bold math symbols
\usepackage{mathtools}

% Figure packages
\usepackage{caption} % Caption formatting for university standard
\usepackage{graphicx} % includegraphics command
\usepackage{subcaption} % Subfigures
\usepackage[section]{placeins} % Place floats in section
\usepackage{wrapfig}

% Table packages
\usepackage{booktabs} % Better tables
\usepackage{bigstrut} % Merged table cells
\usepackage{longtable} % Tables which overflow into next page
\usepackage{array}
\usepackage{colortbl} % Color table cells
\usepackage{makecell}
\usepackage{multirow}

% Fonts
\usepackage{lmodern} % Use latin modern rather than computer modern. Better for font encoding.
\usepackage[T1]{fontenc} % Allow text to be searchable in output

% Other packages
\usepackage{appendix} % Appendix environment
\usepackage{nextpage} % Cleartooddpage command
%\usepackage[square,comma,sort,numbers]{natbib} % Reference formatting
\usepackage{setspace} % Line spacing
\usepackage{listings} % Display code with syntax highlighting
\usepackage{upquote} % Vertical quotes in verbatim
\usepackage{xcolor} % Colors
\usepackage{titlesec} % Header spacing
\usepackage{xparse} % for tcolorbox
\usepackage[listings]{tcolorbox} % Colored boxes for highlighting syntax
\tcbuselibrary{breakable}
\tcbuselibrary{skins}
\usepackage{enumitem} % better enumerate/itemize options
\usepackage{fancyhdr}
\usepackage{multicol}
\usepackage{ifthen}
\usepackage{xstring}

% Table of contents
\usepackage{imakeidx} % Index page
\usepackage{tocloft} % Control of table of contents
\usepackage[nottoc]{tocbibind} % Adds bibliography, table of tables, table of figures, to table of contents
\usepackage[bookmarks,linktocpage,hidelinks]{hyperref} % Hyperlinks for sections, figures, etc.

% Formatting
% Page format
\setlength{\oddsidemargin}{0.00in}  % Left side margin for odd numbered pages
\setlength{\evensidemargin}{0.00in} % Right side margin for even numbered pages
\setlength{\topmargin}{0.00in}      % Top margin
\setlength{\headheight}{0.20in}     % Header height
\setlength{\headsep}{0.20in}        % Separation between header and main text
\setlength{\topskip}{0.00in}        % Top skip
\setlength{\textwidth}{6.50in}      % Width of the text
\setlength{\textheight}{8.50in}     % Height of the text
\setlength{\footskip}{0.50in}       % Foot skip
\setlength{\parindent}{0.00in}      % First line indentation
\setlength{\parskip}{6pt}        % Space between two paragraphs

% Captions (figures, tables, etc.)
\setlength{\floatsep}{\parskip}          % Space left between floats.
\setlength{\textfloatsep}{\floatsep}   % Space between last top float
% or first bottom float and the text
\setlength{\intextsep}{\floatsep}      % Space left on top and bottom
% of an in-text float
\setlength{\abovecaptionskip}{0.1in plus 0.25in}  % Space above caption
\setlength{\belowcaptionskip}{0.1in plus 0.25in}  % Space below caption
\setlength{\captionmargin}{0.50in}     % Left/Right margin for caption
\setlength{\abovedisplayskip}{0.00in plus 0.25in} % Space before Math stuff
\setlength{\belowdisplayskip}{0.00in plus 0.25in} % Space after Math stuff
\setlength{\arraycolsep}{0.10in}       % Gap between columns of an array
\setlength{\jot}{0.10in}                % Gap between multiline equations
\setlength{\itemsep}{0.10in}           % Space between successive items

% Counters (no section numbering)
\setcounter{tocdepth}{3}
\setcounter{secnumdepth}{0}

% Spacing
\setstretch{1.5}

\titlespacing*{\section}{0cm}{6pt}{6pt}[0cm]
\titlespacing*{\subsection}{0cm}{6pt}{6pt}[0cm]
\titlespacing*{\subsubsection}{0cm}{6pt}{6pt}[0cm]

\titleformat{\section}
{\sffamily\huge}{}{0pt}{\titlerule\vspace{-0.2cm}}
\titleformat{\subsection}
{\sffamily\itshape\Large}{}{0pt}{}

% Macro for syntax
\newtcolorbox{syntax}{
    size=small,
    sharp corners,
    colframe=black,
    colback=yellow,
    fontupper=\bfseries\ttfamily
}

% Macro for argument table
\newenvironment{args}{
    \begin{tabular}{>{\bfseries\ttfamily}p{0.25\linewidth} p{0.69\linewidth}}
    }{
    \end{tabular}\par
    \vspace{0.5\baselineskip}
}

% Note: Requires packages "listing", "xcolor", and "textcomp"
\lstdefinelanguage{verbatim}{
    basicstyle=\ttfamily\small,
    xleftmargin=9pt,
    xrightmargin=9pt,
    columns=fullflexible,
    keepspaces=true,
    breaklines=true
}

% Example code
\AtBeginDocument{
\newtcolorbox[blend into=listings]{example}[2][]{
    colback=blue!3!white,
    colframe=black,
    colbacktitle=blue!15!white,
    coltitle=black,
    sharp corners,
    enhanced,
    breakable,
    size=small,
    before upper={
        \setstretch{1.0}\lstset{language=verbatim}\vspace{3pt}\textsf{\textit{Code:}}
    },
    subtitle style={
        colback=blue!20!white,
        fonttitle=\sffamily
    },
    before lower={
        \setstretch{1.0}\lstset{language=verbatim}\vspace{3pt}\textsf{\textit{Output:}}
    },
    fonttitle=\sffamily,
    title={#2},
    #1
}
}

% Links to sub and subsub commands - optional boolean argument, default true. if false, only displays subcmd.

% Commands (and command ensembles)
\newcommand{\command}[1]{\protect\hypertarget{#1}{#1}\index{#1}}
\newcommand{\subcommand}[2]{\protect\hypertarget{#1 #2}{#1 #2}\index{#1!#2}}
\newcommand{\cmdlink}[1]{\protect\hyperlink{#1}{\textit{#1}}}
\newcommand{\subcmdlink}[3][1]{\protect\hyperlink{#2 #3}{\ifnum#1=1\relax\textit{#2 #3}\else\textit{#3}\fi}}

% Methods (first arg is class)
\newcommand{\method}[2]{\protect\hypertarget{$#1Obj #2}{\$#1Obj #2}\index{#1 methods!#2}}
\newcommand{\methodlink}[3][1]{\protect\hyperlink{$#2Obj #3}{\ifnum#1=1\relax\textit{\$#2Obj #3}\else\textit{#3}\fi}}

% Macros for figure/table names
\newcommand{\fig}{\figurename\ }
\newcommand{\figs}{\figurename s }
\newcommand{\tbl}{\tablename\ }
\newcommand{\tbls}{\tablename s }
\newcommand{\eq}{Eq. }
\newcommand{\eqs}{Eqs. }
\renewcommand{\lstlistingname}{Example}% Listing -> Example
\renewcommand{\lstlistlistingname}{List of \lstlistingname s}% List of Listings -> List of Examples
\newcommand{\ex}{Example }
\newcommand{\exs}{Examples }
\newcommand{\var}[1]{\texttt{\textbf{\$#1}}}

% Header/footer
\renewcommand{\headrulewidth}{0pt}

% Changes to hyperlinks (URLs)
\renewcommand\UrlFont{\color{blue}\rmfamily}

% New column type 
% https://tex.stackexchange.com/questions/75717/how-can-i-mix-itemize-and-tabular-environments
\newcolumntype{L}{>{\labelitemi~~}l<{}}
\newcommand{\version}{0.4.4}

\renewcommand{\cleartooddpage}[1][]{\ignorespaces} % single side
\newcommand{\caret}{$^\wedge$}

\title{\Huge Tin: A Tcl Package Manager\\\small Version \version}
\author{Alex Baker\\\small\url{https://github.com/ambaker1/Tin}}
\date{\small\today}
\begin{document}
\maketitle
\begin{abstract}
Tin is a package installer for Tcl. 
With Tin, you can easily install packages directly from GitHub.
\end{abstract}
\clearpage
\section{The Tin List}
Tin installs packages that are in the ``Tin List'', which can be modified in the current session with the commands \cmdlink{tin add} and \cmdlink{tin remove}.
\begin{syntax}
\command{tin add} \$name \$version \$repo \$tag \$file
\end{syntax}
\begin{args}
\$name & Package name. \\
\$version & Package version. \\
\$repo & Github repository URL. \\
\$tag & Github release tag for version.  \\
\$file & Installer file path in repo. 
\end{args}

\begin{syntax}
\command{tin remove} \$name <\$version> <\$repo>
\end{syntax}
\begin{args}
\$name & Package name. \\
\$version & Package version to remove (optional, default all versions). \\
\$repo & Repository to remove (optional, default all repositories).
\end{args}

\begin{example}{Adding a package to the Tin List}
\begin{lstlisting}
package require tin
tin add foo 1.0 https://github.com/username/foo v1.0 install_foo.tcl
\end{lstlisting}
\end{example}
\clearpage
\subsection{Auto Packages}
The commands \cmdlink{tin add} and \cmdlink{tin remove} also have alternative syntax for adding and removing Auto-Tin packages.
An Auto-Tin package is one which has a GitHub repository that has release tags corresponding directly with the package versions, such as ``v1.2.3''. 
To be specific, version release tags must match the following regular expression:
\begin{lstlisting}[language=verbatim]
^v(0|[1-9]\d*)(\.(0|[1-9]\d*))*([ab](0|[1-9]\d*)(\.(0|[1-9]\d*))*)?$
\end{lstlisting}

\begin{syntax}
tin add -auto \$name \$repo \$file <{}<-exact> \$version> <\$reqs ...>
\end{syntax}
\begin{args}
\$name & Package name. \\
\$repo & Github repository URL. \\
\$file & Installer file path in repo. \\
-exact & Option to specify exact version. \\
\$version & Package version. \\
\$reqs ... & Package version requirements, mutually exclusive with -exact option. 
\end{args}

\begin{syntax}
tin remove -auto \$name <\$repo> <\$file>
\end{syntax}
\begin{args}
\$name & Package name. \\
\$repo & Repository to remove (optional, default all repositories). \\
\$file & Installer file path to remove (optional, default all installer files).
\end{args}

Then, if a package is configured as an Auto-Tin package, the Tin List can be populated with versions available for installation with the command \cmdlink{tin fetch}.

\begin{syntax}
\command{tin fetch} <\$name>
\end{syntax}
\begin{args}
\$name & Package name (optional, default ``-all'' fetches for all Auto-Tin packages).
\end{args}

\clearpage
\subsection{The Official Tin List}
The ``tinlist.tcl'' file in the Tin installation initializes the Tin List. 
As of Tin version \version, these are the packages in the official Tin List:
\subsubsection{Auto-Tin Packages}
\begin{tabular}{llll}
Package & Repo & File & Version Requirements \\
\midrule
flytrap & \url{https://github.com/ambaker1/flytrap} & install.tcl & 0- \\
tda & \url{https://github.com/ambaker1/Tda} & install.tcl & 0.1.1- \\
tin & \url{https://github.com/ambaker1/Tin} & install.tcl & 0 \\
tintest & \url{https://github.com/ambaker1/Tin-Test} & install.tcl & 0- \\
vutil & \url{https://github.com/ambaker1/vutil} & install.tcl & 0.1.1- \\
wob & \url{https://github.com/ambaker1/wob} & install.tcl & 0.1.3- \\
\bottomrule
\end{tabular}

\clearpage
\subsection{Saving, Clearing, and Resetting the Tin List}
The state of the Tin List can be saved for future sessions with \cmdlink{tin save}, cleared with \cmdlink{tin clear}, and reset to default or factory settings with \cmdlink{tin reset}. 
Note that \cmdlink{tin save} does not modify the ``tinlist.tcl'' file in the Tin installation. 
Rather, it saves to a hidden user-config file located in the user's home directory.
\begin{syntax}
\command{tin save}
\end{syntax}
\begin{syntax}
\command{tin clear}
\end{syntax}
\begin{syntax}
\command{tin reset} <-hard>
\end{syntax}
\begin{args}
-hard & Option to reset to factory settings.
\end{args}

\begin{example}{Saving changes to the Tin List}
\begin{lstlisting}
tin reset -hard
tin add foo 1.0 https://github.com/username/foo v1.0 install_foo.tcl
tin save
\end{lstlisting}
\tcblower

\textit{"$\sim$/.tinlist.tcl" :}
\begin{lstlisting}
tin add foo 1.0 https://github.com/username/foo v1.0 install_foo.tcl
\end{lstlisting}
\end{example}

\clearpage
\subsection{Accessing the Tin List}

The command \cmdlink{tin get} queries basic information about Tin, and returns blank if the requested entry does exist. 
Similar to \cmdlink{tin add} and \cmdlink{tin remove}, it has two forms, one for querying Tin packages and one for querying Auto-Tin packages. 
Returns a dictionary associated with the supplied arguments.
\begin{syntax}
\command{tin get} \$name <\$version> <\$repo> \\
tin get -auto \$name <\$repo> <\$file>
\end{syntax}
\begin{args}
\$name & Package name. \\
\$version & Package version.  \\
\$repo & Github repository URL. \\
-auto & Option to query Auto-Tin packages. \\
\$file & Installer file path in repo. \\
\end{args}

Additionally, the available packages in the Tin List can be queried with the command \cmdlink{tin packages}, and the available versions for each Tin package can be queried with the command \cmdlink{tin versions}.

\begin{syntax}
\command{tin packages} <-auto> <\$pattern>
\end{syntax}
\begin{args}
-auto & Option to search for Auto-Tin packages. By default searches Tin packages only. \\
\$pattern & Optional ``glob'' pattern, default ``\texttt{*}'', or all packages.
\end{args}

\begin{syntax}
\command{tin versions} \$name <{}<-exact> \$version> <\$reqs ...>
\end{syntax}
\begin{args}
\$name & Package name. \\
-exact & Option to specify exact version. \\
\$version & Package version. \\
\$reqs ... & Package version requirements, mutually exclusive with -exact option.
\end{args}

\clearpage
\section{Installing, Uninstalling, and Upgrading Packages}
The command \cmdlink{tin install} installs packages directly from GitHub, and returns the version installed.
The command \cmdlink{tin depend} installs packages only if they are not installed, and returns the version number installed (useful for installation scripts).
The command \cmdlink{tin installed} returns the package version that is installed and meets the version requirements, or blank if it is not installed.
The command \cmdlink{tin uninstall} uninstalls packages (as long as they are in the Tin List), and returns blank if successful.
The command \cmdlink{tin upgrade} upgrades a package within the major version (for minor and patch upgrades) and returns the version number installed.
\begin{syntax}
\command{tin install} \$name <{}<-exact> \$version> <\$reqs ...>
\end{syntax}
\begin{syntax}
\command{tin depend} \$name <{}<-exact> \$version> <\$reqs ...>
\end{syntax}
\begin{syntax}
\command{tin installed} \$name <{}<-exact> \$version> <\$reqs ...>
\end{syntax}
\begin{syntax}
\command{tin uninstall} \$name <{}<-exact> \$version> <\$reqs ...>
\end{syntax}
\begin{syntax}
\command{tin upgrade} \$name <{}<-exact> \$version> <\$reqs ...>
\end{syntax}
\begin{args}
\$name & Package name. \\
-exact & Option to specify exact version. \\
\$version & Package version. \\
\$reqs ... & Package version requirements, mutually exclusive with -exact option.
\end{args}

\begin{example}{Upgrading Tin, and reloading within current interpreter}
\begin{lstlisting}
# Upgrade Tin
package require tin
tin upgrade tin
# Reload Tin
package forget tin
namespace delete tin
package require tin
\end{lstlisting}
\end{example}


\clearpage
\section{Loading and Importing Packages}
Tin also provides advanced tools for loading and importing packages.
The command \cmdlink{tin require} is similar to the Tcl command \textit{package require}, but with the added feature that if the package is missing, it will try to install it with \cmdlink{tin install}.
The command \cmdlink{tin import} additionally handles most use-cases of \textit{namespace import}. 
Both \cmdlink{tin require} and \cmdlink{tin import} return the version number of the package imported.
\begin{syntax}
\command{tin require} \$name <{}<-exact> \$version> <\$reqs ...>
\end{syntax}
\begin{args}
\$name & Package name. \\
-exact & Option to specify exact version. \\
\$version & Package version. \\
\$reqs ... & Package version requirements, mutually exclusive with -exact option.
\end{args}
\begin{syntax}
\command{tin import} <-force> <\$patterns from> \$name <{}<-exact> \$version> <\$reqs ...> <as \$ns>
\end{syntax}
\begin{args}
-force & Option to overwrite existing commands. \\
\$patterns & Commands to import, or ``glob'' patterns, default ``\texttt{*}'', or all commands. \\
\$name & Package name. \\
-exact & Option to specify exact version. \\
\$version & Package version. \\
\$reqs ... & Package version requirements, mutually exclusive with -exact option. \\
\$ns & Namespace to import into. Default global namespace, or ``\texttt{::}''.
\end{args}

\begin{example}{Importing all commands package ``foo''}
\begin{lstlisting}
package require tin
tin import foo 1.0
\end{lstlisting}
\end{example}

\clearpage
\section{Utilities for Package Development}
In addition to commands for installing and loading packages, Tin provides a few commands intended to help in writing installation and build files for your packages.
\subsection{Creating Package Directories}
The command \cmdlink{tin mkdir} creates a library directory to install a package in, with a normalized naming convention that allows it to be uninstalled easily with \cmdlink{tin uninstall}. 
\begin{syntax}
\command{tin mkdir} <-force> <\$basedir> \$name \$version
\end{syntax}
\begin{args}
-force & Option to create fresh library directory (deletes existing folder). \\
\$basedir & Base directory, default one folder up from the Tcl library folder. \\
\$name & Package name. \\
\$version & Package version.
\end{args}

See the example installation file for a package ``foo'' that requires the package ``bar 1.2'', and installs in library folder ``foo-1.0''.
\begin{example}{Example file ``install\textunderscore{}foo.tcl''}
\begin{lstlisting}
package require tin
tin depend bar 1.2
set dir [tin mkdir -force foo 1.0]
file copy README.md $dir
file copy LICENSE $dir
file copy lib/bar.tcl $dir
file copy lib/pkgIndex.pdf $dir
\end{lstlisting}
\end{example}

\clearpage
\subsection{Building Library Files from Source with Configuration Variable Substitution}
The command \cmdlink{tin bake} takes an input text file, and writes an output text file after substitution of configuration variables such as \texttt{@VERSION@}.
This is especially helpful for ensuring that the package version is consistent across the entire project.

\begin{syntax}
\command{tin bake} \$inFile \$outFile \$config
\end{syntax}
\begin{args}
\$inFile & Source file to read from. \\
\$outFile & File to write to after substitution. \\
\$config & Dictionary of config variable names and values. Config variables must be uppercase alphanumeric.
\end{args}

See below for an example of how \cmdlink{tin bake} can be used to automatically update a ``pkgIndex.tcl'' file:

\begin{example}{Building a ``pkgIndex.tcl'' file}
\begin{lstlisting}
package require tin
tin bake pkgIndex.tin pkgIndex.tcl {VERSION 1.0}
\end{lstlisting}
\tcblower

\textit{"pkgIndex.tin" :}
\begin{lstlisting}
package ifneeded foo @VERSION@ [list source [file join $dir foo.tcl]]
\end{lstlisting}
\textit{"pkgIndex.tcl" :}
\begin{lstlisting}
package ifneeded foo 1.0 [list source [file join $dir foo.tcl]]
\end{lstlisting}
\end{example}
\end{document}



